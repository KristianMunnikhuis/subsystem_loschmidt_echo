\documentclass{article}
\usepackage{graphicx} 
\usepackage{braket}
\usepackage{biblatex}
\usepackage{amsmath}
\begin{document}
\section{Landau-Zener Solution}
The time depednent equations for the TFIM reduce to 

\begin{equation}
    i \hbar \frac{d u_k}{d\tau } = 
    -\frac{1}{2} ( \tau \Delta_k) u_k + \frac{1}{2}v_k
\end{equation}
\begin{equation}
    i \hbar \frac{d v_k}{d\tau } = 
    +\frac{1}{2} ( \tau \Delta_k) v_k + \frac{1}{2}u_k
\end{equation}

Here, $\tau$ depends on the time $t$ and the quench protocol, while $\Delta_k$ is the gap parameter for mode $k$. The functions $u_k(\tau)$ and $v_k(\tau)$ describe the time evolution of the Bogoliubov quasiparticle amplitudes for each momentum mode $k$.

We can define them as follows:

\begin{equation}
    \Delta_k ^{-1} = 4 J \tau_Q \sin^2(ka)
\end{equation}
\begin{equation}
    \tau = 4 J \tau_q \sin(ka)(\frac{t}{\tau_Q}+\cos(ka))
\end{equation}
\end{document}