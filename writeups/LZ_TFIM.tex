\documentclass{article}
\usepackage{graphicx} 
\usepackage{braket}
\usepackage{biblatex}
\usepackage{amsmath}
\begin{document}
\section{Landau-Zener Solution to Kibble-Zurek Scaling}
This section comes directly from "Dynamics of a Quantum Phase Transition: Exact Solution in Quantum Ising model"


The time depednent equations for the TFIM reduce to 

\begin{equation}
    i \hbar \frac{d u_k}{d\tau } = 
    -\frac{1}{2} ( \tau \Delta_k) u_k + \frac{1}{2}v_k
\end{equation}
\begin{equation}
    i \hbar \frac{d v_k}{d\tau } = 
    +\frac{1}{2} ( \tau \Delta_k) v_k + \frac{1}{2}u_k
\end{equation}

Here, $\tau$ depends on the time $t$ and the quench protocol, while $\Delta_k$ is the gap parameter for mode $k$. The functions $u_k(\tau)$ and $v_k(\tau)$ describe the time evolution of the Bogoliubov quasiparticle amplitudes for each momentum mode $k$.

We can define them as follows:

\begin{equation}
    \Delta_k ^{-1} = 4 J \tau_Q \sin^2(ka)
\end{equation}
\begin{equation}
    \tau = 4 J \tau_q \sin(ka)(\frac{t}{\tau_Q}+\cos(ka))
\end{equation}

This can be solved by first recognizing that only modes iwth small energy contribute in the limit of a large quench: 
\begin{equation}
    |ka| < \frac{\pi}{4}
\end{equation}

For these modes we may use the LZ formula for excitation probability:

\begin{equation}
    p_k \approx e^{-\frac{\pi}{2 \hbar \Delta_k}}
\end{equation}

This is valid when

\begin{equation}
    ka = (4 \pi J \tau_Q / \hbar )^{-\frac{1}{2}} << \frac{\pi}{4}
\end{equation}

or equivalently, when 

\begin{equation}
    \tau_Q >> \frac{4\hbar }{\pi^3 J}
\end{equation}

From which we then derive the number of kinks $\mathcal{N}$ or the defect density $$n = \frac{\mathcal{N}}{N}$$

\begin{equation}
    n = \lim_{N \to \infty} \frac{\mathcal{N}}{N} = \frac{1}{2\pi} \int ^{\pi}_{\pi} d(ka) p_k = \frac{1}{2\pi} \frac{1}{\sqrt{2J \tau_Q/\hbar}}
\end{equation}

It should be noted that this calculation of defect density provides $O(1)$ corrections to the formula suggested by Zurek. The real highlight is the prediction of power law scaling of defect density and $\tau_Q$:

$$n \propto \tau_Q^{-\frac{1}{2}}$$

The paper also points out that for a finite chain we can also predict the fastest $\tau_Q$ that no particles get excited (which can be done becuase a finite chain is gapped).

This quantitiy is given as 
\begin{equation}
    \mathcal{P}_{GS} = \prod_{k>0} (1-p_k)
\end{equation}

Only the $\frac{\pi}{N}$ pair is likely to be excited (Question: Why?) so we can approximate that 

\begin{equation}
    \mathcal{P}_{GS} = 1- p_{\frac{\pi}{N}} \approx 1 - \exp(-2 \pi^3 \frac{J \tau_Q}{\hbar N^2})
\end{equation}

So we can estimate that an adiabatic quench occurs when 

$$\tau_Q > \tau_Q^{ad} = \frac{\hbar N^2}{2 \pi^3 J}$$

Inverting this relationship tells us that the size $N$ of a defect free chain grows as $\tau_Q^{\frac{1}{2}}$


\subsection{Ferromagnetic to Paramagnetic}

The previous section was for a disordered to ordered quench. Now, we can analyze the opposite.

We can assume the value of $g$ to be of the form 

$$g(t>0) = \frac{t}{\tau_Q}$$ 

and that when $g=0$ we are in the even parity groundstate.

In the $g>1$ groundstate limit we expect all spins to be in the $+x$ orientation, so we may define ground state fidelity as 

\begin{equation}
    \mathcal{F} = \frac{1}{2}(N - \sum _{n+1}^{N} \sigma^x_n)
\end{equation}

when $g>>1$ the hamiltonian reduces to $$H \approx -Jg \sum^N_{n=1} \sigma^x_n = -Jg(N-2\mathcal{F})$$

At the same time, 

$$H^{+} \approx 2Jg\sum_k \gamma^\dagger_k \gamma_k - JgN$$

so we can recognize via inspection that 

$$\mathcal{F} = \sum_k \gamma_k^\dagger \gamma_k$$

which allows us to predict that defect density is the same, 

\begin{equation}
    f = \frac{1}{2\pi} \frac{1}{\sqrt{2 J \tau_Q /\hbar}}
\end{equation}

\section{Equilibrium Scaling exponents of Projector Operators}

For this section, we will slightly modify the hamiltonian by rotation. 

\begin{equation}
    \hat{H} = -J \sum_{i=1}^{L} \sigma^{x}_{i} \sigma^{x}_{i+1} -h \sum_{i=1}^{L} \sigma^{z}_{i}
\end{equation}

Now we will discuss the scaling exponents of the projector operators. We approach this problem in the even parity sector so odd numbers of spin operators give 0 (e.g. $\braket{\sigma_x}=0$)

For starters, we can look at $P_2$:

\begin{equation}
    \braket{P_2} = \frac{1}{4N} \sum_{i=1}^{L} \braket{1} + \braket({\sigma^x_i \sigma^{x}_{i+1}}) \approx \frac{1}{4}+\frac{ 
        \braket{{\sigma^x_0 \sigma^{x}_{1}}}}{4}
\end{equation}

and how it is expected to scale near the critical point assuming the thermodynamic limit. Also, assuming PBC then the last $\approx \to =$


\end{document}